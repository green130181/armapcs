\chapter{附录1}

\section{arm寄存器介绍}
\newpage
\begin{table}[htbp]
\caption{常用寄存器列表}
\centering
\begin{tabular}{p{20pt}p{20pt}p{20pt}p{100pt}}
    \toprule
    原名& 别名 & 又称 & 解释\\
    \midrule
    r0  &  a1 &      & \\
    r1  &  a2 &      & \\
    r2  &  a3 &      & \\
    r3  &  a4 &      & \\
    r4  &  v1 &      & \\
    r5  &  v2 &      & \\
    r6  &  v3 &      & \\
    r7  &  v4 &      & \\
    r8  &  v5 &      & \\
    r9  &  v6 &   sb & stack base\\
    r10 &  v7 &   sl & stack limit\\
    r11 &  v8 &   fp & frame pointer\\
    r12 &  ip &      & intra-procedure-call scratch register\\
    r13 &  sp &      & stack pointer\\
    r14 &  lr &      & link pointer\\
    r15 &  pc &      & program counter\\
    \bottomrule
\end{tabular}
\end{table}

\section{其他寄存器}

\subsection{Predeclared program status register names}
The following program status register names are predeclared:
\begin{itemize}
    \item cpsr and CPSR (current program status register)
    \item spsr and SPSR (saved program status register).
\end{itemize}

\subsection{Predeclared floating-point register names}
The following floating-point register names are predeclared:
\begin{itemize}
    \item f0-f7 and F0-F7 (FPA registers)
    \item s0-s31 and S0-S31 (VFP single-precision registers)
    \item d0-d15 and D0-D15 (VFP double-precision registers).
\end{itemize}

\subsection{Predeclared coprocessor names}
The following coprocessor names and coprocessor register names are predeclared:
\begin{itemize}
    \item p0-p15 (coprocessors 0-15)
    \item c0-c15 (coprocessor registers 0-15).
\end{itemize}

\section{example}
\subsection{backtrace for gcc 3.4.3}
\begin{code}
struct layout
{
    struct layout *next;        /* ! This is the FP pointing to the previous frame           */
    void *sp;                   /* ! This is the SP pointing to the top of previous frame    */
    void *return_address;       /* ! This is the RA pointing to the location to return       */
    void *pc;                   /* ! This is the PC pointing to the caller function          */
};


# define BOUNDED_N(PTR, N) (PTR)
#define BOUNDED_1(PTR) BOUNDED_N (PTR, 1)
/*  Get some notion of the current stack.  Need not be exactly the top
 *     of the stack, just something somewhere in the current frame.  */
#ifndef CURRENT_STACK_FRAME
# define CURRENT_STACK_FRAME  ({ char __csf; &__csf; })
#endif

/*  By default we assume that the stack grows downward.  */
#ifndef INNER_THAN
# define INNER_THAN <
#endif

/*  By default assume the `next' pointer in struct layout points to the
 *     next struct layout.  */
#ifndef ADVANCE_STACK_FRAME
# define ADVANCE_STACK_FRAME(next) BOUNDED_1 ((struct layout *) ((int)next - 12))
#endif

/*  By default, the frame pointer is just what we get from gcc.  */
#ifndef FIRST_FRAME_POINTER
# define FIRST_FRAME_POINTER  (__builtin_frame_address (0) - 12)
#endif

/*
 * 减12的目的是为了跳过pc lr sp去取上一个栈帧的fp存储的地址
 */
extern void * __libc_stack_end;
    int
\end{code}
\begin{code}
backtrace (array, size)
    void **array;
    int size;
{
    struct layout *current = NULL;
    /* void *top_frame;*/
    void *top_frame = NULL;
    void *top_stack = NULL;
    int cnt = 0;

    top_frame = FIRST_FRAME_POINTER;
    top_stack = CURRENT_STACK_FRAME;

    /*  We skip the call to this function, it makes no sense to record it.  */
    current = BOUNDED_1 ((struct layout *) top_frame);
    while (cnt < size)
    {
        if ((void *) current INNER_THAN top_stack
                || !((void *) current INNER_THAN __libc_stack_end))
            /*  This means the address is out of range.  Note that for the
             *  toplevel we see a frame pointer with value NULL which clearly is
             *  out of range.  */
            break;

        array[cnt++] = current->return_address;

        current = ADVANCE_STACK_FRAME (current->next);
    }

    return cnt;
}
\end{code}

\subsection{backtrace for gcc 4.4.1}
\begin{code}
struct layout
{
    void *fp;
    void *lr;
};


# define BOUNDED_N(PTR, N) (PTR)
#define BOUNDED_1(PTR) BOUNDED_N (PTR, 1)
/* Get some notion of the current stack.  Need not be exactly the top
   of the stack, just something somewhere in the current frame.  */
#ifndef CURRENT_STACK_FRAME
# define CURRENT_STACK_FRAME  ({ char __csf; &__csf; })
#endif

/* By default we assume that the stack grows downward.  */
#ifndef INNER_THAN
# define INNER_THAN <
#endif

/* By default assume the `next' pointer in struct layout points to the
   next struct layout.  */
#ifndef ADVANCE_STACK_FRAME
# define ADVANCE_STACK_FRAME(next) BOUNDED_1 ((struct layout *) (((int)next) - 4))
#endif

/* By default, the frame pointer is just what we get from gcc.  */
#ifndef FIRST_FRAME_POINTER
# define FIRST_FRAME_POINTER  (__builtin_frame_address (0) - 4)
#endif

extern void * __libc_stack_end;
int backtrace (void **array, int size)
{
    struct layout *current;
    void *top_frame;
    void *top_stack;
    int cnt = 0;

    top_frame = FIRST_FRAME_POINTER;
    top_stack = CURRENT_STACK_FRAME;
\end{code}

\begin{code}
    /* We skip the call to this function, it makes no sense to record it.  */
    current = BOUNDED_1 ((struct layout *) top_frame);
    while (cnt < size)
    {
        if ((void *) current->fp INNER_THAN top_stack
                || !((void *) current->fp INNER_THAN __libc_stack_end))
        {
            /* This means the address is out of range.  Note that for the
               toplevel we see a frame pointer with value NULL which clearly is
               out of range.  */
            break;
        }

        array[cnt] = current->lr;
        cnt++;

        current = ADVANCE_STACK_FRAME (current->fp);
    }
    /*
     * In leaf function
     * gcc not push lr when call it
     */
    printf(" ");
    return cnt;
}
\end{code}

\subsection{test case for backtrace}
\begin{code}
int f2(int a,int b)
{
    int j, nptrs;
    void *buffer[100];
    nptrs = backtrace(buffer, 100);
    printf("file:%s, line:%d, nptrs:%d\n", __FILE__, __LINE__, nptrs);
    for (j = 0; j < nptrs; j++)
        printf("%p\n", buffer[j]);
    return v;
}

int f1(int a,int b)
{
    f2(a,b);
    return 0;
}
int main(int argc, char *argv[])
{
    f1(2,3);
    return 0;
}
\end{code}

\subsection{backtrace tsk\_struct for gcc 4.4.1 in kernel}
主要区别在于这个在内核态已经可以看到用户态的任务栈,
找到当前的sp fp 倒出线程栈
\begin{code}
/*
 * Something tried to access memory that isn't in our memory map..
 * User mode accesses just cause a SIGSEGV
 */
static void
__do_user_fault(struct task_struct *tsk, unsigned long addr,
		unsigned int fsr, unsigned int sig, int code,
		struct pt_regs *regs)
{
	struct siginfo si;

#ifdef CONFIG_DEBUG_USER
	if (user_debug & UDBG_SEGV) {
		printk(KERN_DEBUG "%s: unhandled page fault (%d) at 0x%08lx, code 0x%03x\n",
		       tsk->comm, sig, addr, fsr);
		show_pte(tsk->mm, addr);
		show_regs(regs);
	}
#endif
    /*
     *  添加的代码
     */
    my_show_stack(current, (unsigned long *)regs->ARM_sp, regs, addr, fsr);

	tsk->thread.address = addr;
	tsk->thread.error_code = fsr;
	tsk->thread.trap_no = 14;
	si.si_signo = sig;
	si.si_errno = 0;
	si.si_code = code;
	si.si_addr = (void __user *)addr;
	force_sig_info(sig, &si, tsk);
}
\end{code}


\begin{code}
void my_show_stack(struct task_struct *tsk, unsigned long *sp,struct pt_regs *regs, unsigned long addr, unsigned int fsr)
{
    int i;
    int stack_len = 0;
    int upload_times = 0, remain_bytes = 0;
    unsigned long sp_addr = 0;
    struct vm_area_struct *vma = NULL;
    struct u_stack_packet info;

    if ( NULL == tsk || NULL == sp || NULL == regs)
    {
        return;
    }
    
    print_reg(tsk, regs, addr, fsr);
}
\end{code}

\begin{code}
struct layout
{
    void *fp;
    void *lr;
};


# define BOUNDED_N(PTR, N) (PTR)
#define BOUNDED_1(PTR) BOUNDED_N (PTR, 1)
/* Get some notion of the current stack.  Need not be exactly the top
   of the stack, just something somewhere in the current frame.  */
#ifndef CURRENT_STACK_FRAME
# define CURRENT_STACK_FRAME  (regs->ARM_sp)
#endif

/* By default we assume that the stack grows downward.  */
#ifndef INNER_THAN
# define INNER_THAN <
#endif

/* By default assume the `next' pointer in struct layout points to the
   next struct layout.  */
#ifndef ADVANCE_STACK_FRAME
# define ADVANCE_STACK_FRAME(next) BOUNDED_1 ((struct layout *) (((int)next) - 4))
#endif

/* By default, the frame pointer is just what we get from gcc.  */
#ifndef FIRST_FRAME_POINTER
# define FIRST_FRAME_POINTER  (regs->ARM_fp - 4)
#endif

void print_reg(struct task_struct *tsk, struct pt_regs *regs, 
    unsigned long addr, unsigned int fsr)
{
    int i = 0, j = 0;
    int reg_num = 0;
    printk("+++++++++++++++++++++++++++++++++++++++++++++++++++++++++++++++++++++++++++++++\n");
    printk("++  %s is dead (PID: %d)\n", tsk->comm, tsk->pid);
    printk("+++++++++++++++++++++++++++++++++++++++++++++++++++++++++++++++++++++++++++++++\n");

\end{code}
\begin{code}
    printk("\n");
    printk("SP = 0x%08x  FP = 0x%08x  IP = 0x%08x  LR = 0x%08x  PC = 0x%08x\n", 
        (unsigned int)regs->ARM_sp, (unsigned int)regs->ARM_fp, 
        (unsigned int)regs->ARM_ip, (unsigned int)regs->ARM_lr, 
        (unsigned int)regs->ARM_pc);

    printk("CPSR = 0x%08x\n", (unsigned int)regs->ARM_cpsr);
    if (fsr & (1 << 11))    /* write? */
        printk("access: [Write], addr = 0x%08x \n", (unsigned int)addr);
    else
        printk("access: [Read], addr = 0x%08x \n", (unsigned int)addr);

    printk("\n");
    
    for (i = 0; i < 6; i++)
    {
        for (j = 0; j < 3; j++)
        {
            reg_num = i * 3 + j;
            printk("r%d = 0x%08x  ", reg_num, (unsigned int)regs->uregs[reg_num]);
        }
        printk("\n");
    }
    printk("\n");
    printk("start backtrace task +++++++++++++++++++++++++++++++++++\n");
    struct vm_area_struct *vma = NULL;
    vma = find_vma(tsk->mm, regs->ARM_sp);
    if (NULL == vma)
        return;

\end{code}
\begin{code}
    printk("file:%s, line:%d, ARM_sp:%p, vma->vm_start:%p, vma->vm_end:%p\n", 
            __FILE__, __LINE__, regs->ARM_sp, vma->vm_start, vma->vm_end);
    if ((regs->ARM_sp > vma->vm_start) && (regs->ARM_sp < vma->vm_end))
    {
        struct layout *currentstruct;
        void *top_frame;
        void *top_stack;

        top_frame = FIRST_FRAME_POINTER;
        top_stack = CURRENT_STACK_FRAME;

        /* We skip the call to this function, it makes no sense to record it.  */
        currentstruct = BOUNDED_1 ((struct layout *) top_frame);
        while (1)
        {
            if ((void *) currentstruct->fp INNER_THAN top_stack
                    || !((void *) currentstruct->fp INNER_THAN vma->vm_end))
            {
                /* This means the address is out of range.  Note that for the
                   toplevel we see a frame pointer with value NULL which clearly is
                   out of range.  */
                printk("file:%s, line:%d, currentstruct->fp:%p, top_stack:%p, \
                        vma->vm_end:%p\n", 
                        __FILE__, __LINE__, currentstruct->fp, 
                        top_stack, vma->vm_end);
                break;
            }
            printk("currentstruct->lr:%p\n", currentstruct->lr);
            currentstruct = ADVANCE_STACK_FRAME (currentstruct->fp);
        }
    }
    printk("end   backtrace task ++++++++++++++++++++\n");
    printk("\n\n");

    printk("\n");
    printk("+++++++++++++++++++++++++++++\n");
    printk("\n\n");
}
\end{code}


\subsection{backtrace 调试方法}
\begin{itemize}
    \item 要读懂该体系结构的汇编
    \item 先编译个带函数调用的小程序,objdump反汇编,\
          然后直接查看汇编中的压栈过程,反向倒回来即可
    \item 编写过程中可以借助gdb 的examine命令(简写是x)\
          查看内存进行调试
\end{itemize}

